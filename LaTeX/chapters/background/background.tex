%background
\documentclass[../../main/main.tex]{subfiles}


\begin{document}
\title{Supplemental Material}

\chapter{Background}\label{chp:background}
This appendix provides some supplemental information that may be helpful to the reader.  It discusses formal methods, functional programming, algebraic data types in ML, the Higher Order Logic (\gls{hol}) Interactive Theorem Prover, other theorem provers, and how to compile the LaTeX and \gls{hol} files included with this master thesis.

\section{Formal Methods}
(Primary source for this section is \cite{formalCarnegie})

Formal methods improve the reliability and correctness of systems \cite{formalmethodslcarke}.  They are particularly useful in the design phase of systems engineering.  But, they are also employed to varying degrees throughout all aspects of the systems engineering process\footnote{For example, there is notable progress being made in formally verifying c code.  See for example \cite{jvmcmv} and \cite{VARVEL}}.  

Formal methods analysis is a three phase process: (1) formal specification of the system using a modeling language, (2) verification of the specification, and (3) implementation \cite{formalCarnegie}.  Specification consists of describing the system in a mathematical-based modeling language \cite{formalCarnegie}.  Verification entails proving properties of the system, typically using either model checking or theorem proving techniques.  Implementation depends on the type of system (i.e., software, hardware, human-centered system, etc.).

The two most-noted formal methods are model checking and theorem proving.  Model checking involves testing possible states of a system for correctness.  Model checking is touted more as an error checking tool.  It exhausts all possible states (or test states) in search of failures.  But, the absence of failure is not proof of correctness.  Furthermore, for large systems, model checking is resource intensive.  It is subject to the state explosion problem, wherein the number of states of the system grows exponentially with the complexity of the system \cite{stateexplosion}. 

Theorem proving, on the other hand, employs a formal logic to prove properties of the system.  Formal proofs remain true for any test case \cite{formalCarnegie}.  This means that these are proofs of correctness and not just proofs of the absence of failure.  Theorem proving is usually partially or fully automated \cite{wikiformalmethods}.  It often requires specialized knowledge and a sufficient degree of competency in mathematics\footnote{and a good amount of patience, in the author's opinion.}. However, formal theorem proving techniques have been successfully taught to undergraduates\footnote{The PI Professor Shiu-Kai Chin teaches CSBD, which includes theorem proving, to both undergraduates and graduates at Syracuse University.}.  


Model checking and theorem proving often work synergistically.  Both methods offer benefits that the other does not.  As a whole, they improve the reliability of the system's design.   As an example, ElasticSearch successfully applies both methods to its search methodology on distributed systems \cite{elasticsearch}.  

Formal methods are used in some areas of systems engineering more than others.  Some engineers are reluctant to use them because of the additional work and level of expertise required.  However, when correctness is non-negotiable, such as safety and security, formal methods become essential.  Formal methods are predicted to increase in usage as tools become easier to use and engineering curricula increasingly offer formal methods as part of their core\cite{formalCarnegie}.

Specification has its benefits, not only as a precursor to verification, but also as a tool for understanding.  Even in cases wherein properties of the system are not proved formally, the act of formally specifying the system adds a degree of rigor to the process. This rigor brings new insights about the system because it causes the engineer to think more systematically about the system's design.  In addition, the conversion of a field's jargon into a precise specification language also aids in reproducibility and communicability\footnote{The later ideas are not specifically stated in the main cite, but follow logically from the author's perspective.} \cite{formalCarnegie}.  

This master thesis applies formal verification methods, specifically theorem proving, to prove the security properties of a system.  Theorem proving is partially automated using the Higher Order Logic (\gls{hol}) Interactive Theorem Prover. 

\section{Functional Programming}
(Primary source for this section is \cite{functionalprogramming})

Functional programing is a style of programming that uses functions to define program behavior.  Haskell and ML or polyML\footnote{ML is not considered a purely functional language.  But, it's pretty close.  Even Haskell, which is the archetypical functional programming language, has non-pure components, in particular IO operations.} (meta language) are examples of functional programming languages.  Functional programing is inherently different than procedural or object-oriented programming, which use procedures or objects and classes to define program behavior.  C  and Pascal are examples of procedural programming languages.  C++ and Java are examples of object-oriented programming languages.    

Functional programming languages are thought to be more pure.  They have fewer side effects than procedural or object-oriented programming.  They produce fewer bugs.  Functional programming languages are thus considered more reliable. This is why theorem provers are typically implemented in functional programming languages.  

This master thesis relies on the Higher Order Logic (\gls{hol}) Interactive theorem prover.  \gls{hol} is implemented in a functional programming language (polyML).  \gls{hol} is thought to be very reliable and trustworthy.

\section{Algebraic Data Types in ML}\label{adt}\label{sec:adtinml}
The primary source for this section is \textit{Introduction to Programming Languages/Algebraic  Data Types} \cite{types}.  

The online book \textit{Introduction to Programming Languages/Algebraic  Data Types} describes types as sets.  For example, the boolean type is a set composed of two values "True" and "False".  

An \gls{adt} is a composite data type.  The boolean type is also an \gls{adt}.  In ML, \gls{adt} are preceded by the \texttt{datatype} keyword.  The boolean data type is defined in figure \ref{booladt}.
  

\begin{figure}[h]
\centering
\includegraphics[width=0.4\textwidth]{../figures/booladt}
\caption{\label{booladt} An implementation of the boolean datatype in ML.  Image from \textit{Introduction to Programming Languages/Algebraic Data Types} \cite{types} }.  
\end{figure}

The boolean datatype can be thought of as the union of the the two singleton sets "True" and "False".  The "\textbar" symbol denotes union in this case.  

An example of a more complicated datatype is the tree datatype defined below.

\begin{figure}[h]
\centering
\includegraphics[width=0.6\textwidth]{../figures/tree}
\caption{\label{tree} An implementation of a tree datatype in ML.  Image from \textit{Introduction to Programming Languages/Algebraic Data Types} \cite{types} }.  
\end{figure}

The first element of the tree datatype is "Leaf".  The second element is a "Node".  The node has three components: tree, int, and another tree.  The "of" keyword in ML indicates that what follows is a type.  Thus, this is a "Node" \textbf{of} type tree * int * tree.  The * symbol represents the cartesian product.  The easiest way to think of this particular example in ML is that the "Node" type is a three-tuple consisting of (tree, int, tree).

In ML,\footnote{Standard ML according to the author \cite{constructor}.} a type constructor is thought of as a function that takes a type as its parameter and returns a type.  In the declaration \texttt{Node of 'a tree * 'a * 'a tree}. \texttt{Node} is a constructor that behaves like a function which returns something of type \texttt{'a tree} \cite{constructor}.


\glsentryshortpl{adt} exhibit the property of pattern matching.  This is typical of functional programming languages.  For example, consider the datatype \textit{weekday} shown below.

\begin{figure}[h]
\centering
\includegraphics[width=0.5\textwidth]{../figures/weekday}
\caption{\label{weekday} An implementation of the weekday datatype in ML.  Image from \textit{Introduction to Programming Languages/Algebraic Data Types} \cite{types} }.  
\end{figure}

With pattern matching, it is possible to define a function \texttt{is_weekend} which returns "True" if it is a weekend and "False" otherwise.  This is shown in figure \ref{isweekend}.
%Consider the following definition for a function that takes one argument (or parameter) of type "weekday."

\begin{figure}[h]
\centering
\includegraphics[width=0.5\textwidth]{../figures/isweekend}
\caption{\label{isweekend} An implementation of the function \textit{is_weekend} in ML.  Image from \textit{Introduction to Programming Languages/Algebraic Data Types} \cite{types} }.  
\end{figure}


The keyword \texttt{fun} must precede any function definition in ML. Next, is the name of the function, in this case \texttt{is_weekend}.  The next word "Sunday" is an element of the datatype \texttt{weekday}.  When ML evaluates the \texttt{is_weekend} function, it will check the argument.  If the argument is "Sunday" then ML will return "True", otherwise it will check the next line.  If that argument equals "Saturday" ML will return "True", otherwise it will check the next line.  The next line contains the "_" character.  In ML, this is similar to the wildcard "*" character in Unix.  In this case, the underscore character tells ML to return "False" for any argument.  The novelty of pattern matching is that function evaluation proceeds in order, allowing for an easy and clean way to define an if-then-else evaluation.

A second property of \glsentryshortpl{adt} is parameterization.  Consider a more general definition of the tree datatype shown below.

\begin{figure}[h]
\centering
\includegraphics[width=0.6\textwidth]{../figures/treeparam}
\caption{\label{treeparam} An implementation of a parametrized tree datatype in ML.  Image from \textit{Introduction to Programming Languages/Algebraic Data Types} \cite{types} }.  
\end{figure}

In this definition, the datatype definition is preceded by the type variable \texttt{'a}.  To declare something of type \texttt{'a Tree}, the \texttt{'a} is instantiated.  For example, \texttt{int tree}, produces the same tree as in figure \ref{tree}.  But, it can also produce a tree of characters with the declaration \texttt{char tree}.  

\texttt{'a} is called a polymorphic type or type variable.  A type variable is just a place holder for some other type or type variable.  Type variables are always preceded with a forward tick mark (apostrophe) in ML.


\section{Higher Order Logic (HOL) Interactive Theorem Prover}\label{app:hol}
The Higher Order Logic (\gls{hol}) Interactive theorem prover is a proof assistant.  \gls{hol} has proved to be a very reliable.  It is widely trusted in the interactive theorem proving community.

At its core, \gls{hol} implements a small set of axioms and a formal logic.  All inferences and theorems must be derived from this small set of axioms using the formal logic.  Reasoning logically with a small set of axioms contributes to the trustworthiness of the system.  The user only has to trust the small set of axioms and the logic (in addition to \gls{hol}'s implementation).  Beyond the competence of the programmer, it is said that if it can't be proved in \gls{hol} then it can be proved.  

\gls{hol} is a strongly-typed system.  This means that data has a predefined type.  As in all functional programming languages, the type of these data can not change.  This adds to the reliability of \gls{hol} by preventing side-effects. \gls{hol} has several built-in data types.  But, the user can also define her own data type.   In addition to datatypes, the user can define her own set of axioms and definitions.  

With user-defined types, axioms, and definitions, the user can describe a system in \gls{hol} and then use \gls{hol}'s formal logic to prove properties of this system.  This is the basis for theorem proving in formal methods.

The version of \gls{hol} used for this master thesis is HOL4.  HOL4 is free software that is BSD licensed \cite{BSDlicenses} and is descended from HOL88 cite{HOL}.  Download instructions can be found at the \gls{hol} website hol-theorem-prover.org \cite{HOL}.

\section{Other Interactive Theorem Provers}
It should be noted that \gls{hol} is not unique. There are other interactive theorem provers on the market. Each has its own niche and dedicated user base.  Choice of theorem prover is typically guided by personal preference and familiarity. The later follows from the somewhat steep learning curve for theorem provers. 

For example, Isabelle/HOL is another popular interactive theorem prover.  The access-control logic (\gls{acl}) has been partially implemented in Isabelle/HOL by Scott Constable, a PhD student in the College of Engineering and Computer Science at Syracuse University.  

\section{How to Compile The Included Files}
All the files necessary to compile the theories discussed in this mater thesis are included.  A diagram of the folder structure is included in the appendix \ref{foldermap}.  To compile the theories and the LaTeX files go to the folder titled MasterThesis.  Open up a terminal.  Type \emph{make} and then hit enter.  Note that to compile the theories HOL and LaTeX must be installed.  


\end{document}